以下为第一章小测的后两题解答,由于涉及较多课上细节,大家普遍答得不太好。{\color{red}所以这启示大家以后一定要好好听讲呀。}

\begin{enumerate}[label=1.\Alph*]
\item
\emph{试通过计算具体说明人们通常称光子的能量表达式$\epsilon=h\nu$为普朗克光量子假设,也称为爱因斯坦光量子理论。爱因斯坦的高明之处何在?}

【简解】

普朗克的推导基于量子假设:引起辐射的谐振子的能量只能取特殊的分立值,故辐射场能量的变化只能是最小能量$\epsilon_0=h\nu$的整数倍。
由此,他通过统计学的玻尔兹曼分布推导出黑体辐射中的普朗克公式$r_B(\lambda,\,T) = \frac{2\pi hc^2}{\lambda^5} \frac{1}{e^{\frac{hc}{\lambda k_B T}}-1}$,
发现与实验高度符合。
因此就逻辑而言,是先基于量子假设而后进行演绎,得到了符合实验的结果,故称为量子假说;此外,普朗克并没有进一步得出光子场本身的能量只能是一份一份的结论,是爱因斯坦做了后续补充。

爱因斯坦的推导没有假设,仅由统计学及描述经典辐射规律的维恩公式得到熵变的表达式$\Delta S=\frac{k_B E}{h\nu}\ln\frac{V}{V_0}$,通过与理想气体熵的表达式做类比,推出光子的能量形式为$\epsilon=h\nu$,说明光子场本身是由不连续的能量单位组成的。这里$\epsilon=h\nu$是推导出的结论,可称为光量子理论。

{\color{red}以上需要补充详细的计算过程。}普朗克的推导请见PPT版讲义第4页;爱因斯坦的推导请见第20页附录。(评分:基础$2'$ + 爱因斯坦推导$3'$ + $S$表达式$2'$ + 与理想气体类比得结论$2'$ + 对普朗克的说明$1'$)


\item
\emph{试证明通常的波函数$\psi(x)$具有$\dv{\ln\psi}{x}$连续的性质,并有朗斯基式。}

考虑$\psi(x)$满足一维定态薛定谔方程
\[-\frac{\hbar^2}{2m}\psi''(x) + V(x)\psi(x) = E\psi(x).\]
因其二阶导数存在,故$\psi(x),\,\psi'(x)$连续,因此在$\psi(x)\neq 0$的分段开区间内,$\frac{\psi'(x)}{\psi(x)}$连续。整理后即得到$\dv{\ln\psi(x)}{x}$连续。

考虑方程的两个解$\psi_1(x),\,\psi_2(x)$。分别代入上式中成立,记为(1), (2)。则计算$\psi_2\times(1)-\psi_1\times(2)$得到
\[\psi_1''(x)\psi_2(x) - \psi_2''(x)\psi_1(x) = 0,\]
整理为
\[\dv{x}(\psi_1'(x)\psi_2(x) - \psi_2'(x)\psi_1(x))=0.\]
因此朗斯基式
\[W(x) = 
\begin{vmatrix}
\psi_1'(x) & \psi_2'(x)\\
\psi_1(x) & \psi_2(x)\\
\end{vmatrix} = 
\psi_1'(x)\psi_2(x) - \psi_2'(x)\psi_1(x) = \text{常数}.\]
考虑一般条件下波函数的边界条件:$x\rightarrow \pm \infty$时$\psi(x)\rightarrow 0$, $\psi'(x)\rightarrow 0$,得到朗斯基式$W(x)=0$.

{\color{red} 评价:这表示着一维体系下满足定态薛定谔方程的所有满足$\eval{\psi}_{x=\pm \infty} = 0$
的解一定都是线性相关的,也即:在特定能量取值下不存在简并态。但在二维或三维情形下上述推导便不再成立(为什么?),且能量简并态成为可能(如氢原子模型中所有主量子数$n$相同的态均为能量简并态)。}

注:部分同学利用$\psi(x),\, \psi^*(x)$满足薛定谔方程表达式,按上述推导得出
\[\begin{vmatrix}
\psi'(x) & \psi^*'(x)\\
\psi(x) & \psi^*(x)\\
\end{vmatrix} = 
\psi'(x)\psi^*(x) - \psi^*'(x)\psi(x) = 0.\]
这固然是成立的,不过它只是朗斯基式的一种特例,只讨论了$\psi(x)$与$\psi^*(x)$的线性相关性。

(评分:证连续$4'$ + 朗斯基正确表达式$2'$ + 薛方程推朗斯基式$3'$ + 讨论边界$1'$)

\end{enumerate}