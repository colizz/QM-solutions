\section{第三、四、五章小测补充题解答}
% \begin{center}
%     \Large{\textbf{第三、四、五章小测解答}}
% \end{center}

角动量算符的对易关系:$\zkh{q_i,\,l_j}=\ihb\eijk q_k\quad\zkh{\vec{q}^2,\,l_i}=0$,其中$q$可以代表$r$,$p$或$l$。

常用恒等式:

$\ee{\alpha\hA}\hB\ee{-\alpha\hA}=\sum_{n=0}^\infty\frac{\alpha^n}{n!}(\mathcal{L}_{\hA})^n(\hB)$,其中$\mathcal{L}_{\hA}(\hX)=\zkh{\hA,\,\hX}$

$\ee{\hA+\hB}=\ee{\hA}\ee{\hB}\ee{-\frac{1}{2}\hC}$,其中$\hC=\zkh{\hA,\,\hB}$,且$\hC$与$\hA$和$\hB$均对易

\begin{enumerate}[label=\textbf{3.\Alph*}, listparindent=\parindent, leftmargin=-0.5mm]

\setcounter{enumi}{5}
\item \emph{题目:定义产生消灭算符
\alg{\ha^\dagger = \sqrt{\frac{1}{2}}\qty(\sqrt{\frac{m\omega}{\hbar}}\hat{x} - \im\sqrt{\frac{1}{m\hbar\omega}}\hat{p})\\
\ha = \sqrt{\frac{1}{2}}\qty(\sqrt{\frac{m\omega}{\hbar}}\hat{x}+ \im\sqrt{\frac{1}{m\hbar\omega}}\hat{p})}
证明:
\begin{enumerate}
    \item 相干态
    \alg{\ket{c}=\ee{c\ha\hcj-c\cj \ha}\ket{0}}是$\ha$的归一化的本征态,本征值为$c$,其中$c$是$\ha^\dagger\ha$的本征态,本征值为$0$.
    \item 相干态$\ket{c}$中坐标和动量满足
    \alg{\sqrt{\jkh{\Delta x^2}\jkh{\Delta p^2}}=\frac{\hbar}{2}.}
\end{enumerate}
}
\noindent【解答】

\begin{enumerate}
    \item 由于$\zkh{\ha,\,\ha\hcj}=1$为常数,可利用恒等式$\ee{\hA+\hB}=\ee{\hA}\ee{\hB}\ee{-\frac{1}{2}\hC}$得到
    \alg{\ee{c\ha\hcj-c\cj \ha}=\ee{-\frac{\abs{c}^2}{2}}\ee{ca\hcj}\ee{-c\cj a}}
    又由于$\ha\ket{0}=0$,可知$\ee{-c\cj \ha}\ket{0}=\ket{0}$,因此
    \alg{\ket{c}&=\ee{c\ha\hcj-c\cj \ha}\\
    &=\ee{-\frac{\abs{c}^2}{2}}\ee{c\ha\hcj}\ket{0}\\
    &=\ee{-\frac{\abs{c}^2}{2}}\sum_{n=0}^\infty\frac{c^n}{\sqrt{n!}}\ket{n}}
    容易看出$\ket{c}$是归一化的,因为
    \alg{\bra{c}\ket{c}=\ee{-\abs{c}^2}\sum_{n=0}^\infty\frac{\abs{c^2}^n}{n!}=\ee{-\abs{c^2}}\ee{\abs{c^2}}=1}
    这里用到了$\ket{n}$这组基正交归一的性质。下面证明$\ket{c}$是$\ha$的本征态:
    \alg{\ha\ket{c}&=\ee{-\frac{\abs{c}^2}{2}}\sum_{n=0}^\infty\frac{c^{n}}{\sqrt{n!}}\sqrt{n}\ket{n-1}\\
    &=\ee{-\frac{\abs{c}^2}{2}}\sum_{n=1}^\infty c\frac{c^{n-1}}{\sqrt{(n-1)!}}\ket{n-1}\\
    &=c\ket{c}}
    可见$\ket{c}$是$\ha$的本征态,本征值是$c$.
    
    \item 首先将$x$和$p$用产生消灭算符表示:
    \alg{&x=\sqrt{\frac{\hbar}{2m\omega}}\xkh{\ha\hcj+\ha}\\
    &p=\im\sqrt{\frac{\hbar m\omega}{2}}\xkh{\ha\hcj-\ha}\\
    &x^2=\frac{\hbar}{2m\omega}\xkh{2\ha\hcj \ha+1+\ha\hcjsq+\ha^2}\\
    &p^2=\frac{1}{2}\hbar m\omega\xkh{2\ha\hcj \ha+1-\ha\hcjsq-a^2}}
    由于$\ket{c}$是$a$的本征态,容易得到$\jkh{\ha}=c$,$\jkh{\ha^2}=c^2$,以及
    \alg{&\jkh{\ha\hcj}=\xkh{\ket{c},\,\ha\hcj\ket{c}}=\xkh{\ha\ket{c},\,\ket{c}}=c\cj\xkh{\ket{c},\,\ket{c}}=c\cj\\
    &\jkh{\ha\hcj \ha}=\xkh{\ket{c},\,\ha\hcj \ha\ket{c}}=\xkh{\ha\ket{c},\,\ha\ket{c}}=c\cj c\xkh{\ket{c},\,\ket{c}}=\abs{c}^2\\
    &\jkh{\ha\hcjsq}={c\cj}^2}
    其中$\xkh{\ket{i},\,\ket{j}}$表示内积。容易看出
    \alg{&\jkh{x}=\sqrt{\frac{\hbar}{2m\omega}}\xkh{c+c\cj}\\
    &\jkh{p}=\im\sqrt{\frac{\hbar m\omega}{2}}\xkh{c\cj-c}\\
    &\jkh{x^2}=\frac{\hbar}{2m\omega}\xkh{2\abs{c}^2+1+c^2+{c\cj}^2}\\
    &\jkh{p^2}=\frac{\hbar m\omega}{2}\xkh{2\abs{c}^2+1-c^2-{c\cj}^2}}
    由此可得
    \alg{\Delta x\Delta p=\sqrt{\xkh{\jkh{x^2}-\jkh{x}^2}\xkh{\jkh{p^2}-\jkh{p}^2}}=\frac{\hbar}{2}}
    
\end{enumerate}
\end{enumerate}


\begin{enumerate}[label=\textbf{4.\Alph*}, listparindent=\parindent, leftmargin=-0.5mm]
\setcounter{enumi}{2}
\item \emph{题目:曾书3.26题,将势能函数改为
\alg{V(x)=\begin{cases}\gamma\,\delta(x), &x< a\\\infty,&x\geq a\end{cases}}
其中$a>0$.}

\noindent【解法一】

采用常规方法计算。首先分析出波函数$\psi(a)=0$, 可分三个区间设波函数为
\alg{\psi(x)=\begin{cases}
\ee{\im k x} + r\ee{-\im k x},& x<0\\
A\sin k(x-a), &0\leq x< a\\
0,& x\geq a
\end{cases}}
待定参量为$r$和$A$。
根据$x=0,\,a$的边条件有
\alg{&1+r = -A\sin ka\\
&\im k(1-r) + \frac{2m\gamma}{\hbar^2}(1+r)=k A\cos ka}
按照之前单$\delta$势无量纲化的方案,设$\theta = \frac{k\hbar^2}{m\gamma}$,
则第二个等式化简为
\[\im\theta (1-r)+2(1+r)=\theta A\cos ka\]
与第一个等式相除,可以化简得到
\[r = \frac{-\im\theta-2-\theta\cot ka}{-\im\theta+2+\theta\cot ka}
= \frac{-\theta \cos ka + (-2-\im\theta)\sin ka}{\theta \cos ka + (2-\im\theta)\sin ka}
= \frac{\ee{-\im ka}+(-1-\im\theta)\,\ee{\im ka}}{(-1+\im\theta)\,\ee{-\im ka}+\ee{\im ka}}\]
上面三种结果是等价表述。显然地,反射系数为$R=\abs{r}^2=1$,透射系数$T=1-R=0$。不存在完全透射即$T=1$的情况。

\noindent{\color{red}\textbf{说明:}}本题的透射系数是针对整个体系而言的,即应考虑$x>a$区域内透射波的情况(显然无透射波存在)。部分同学考虑了经过$\delta(x)$的“透射系数”,但这样其实没有较好的定义,因为$x<0$和$0<x<a$区间内都含有沿两个方向传递的波的成分。抱歉本题没说清楚,大家只要正确计算出上面的$r$就算对了。

\noindent【解法二】

可参考4.26、4.27作业解答里给出的特殊解法。考虑入射波$\ee{\im kx}$,让该波在$\delta$势和墙壁之间无限次反射、透射,可得全空间波的分布,请同学们自己画示意图。需要注意,这里的墙壁会给波带来“半波损失”的效果,即入射墙壁的波前振幅为$\tilde{A}$的话,反射波前振幅应为$-\tilde{A}$,这样才能使二者在墙壁处叠加为0.

由4.26知单$\delta$势对波的反射与透射有$r_0=\frac{1}{\im\theta-1}$, $s_0=\frac{\im\theta}{\im\theta-1}$,将所有$x<0$区域沿$-x$方向传播的波的波前叠加,得到
\alg{r &= r_0-s_0^2\ee{2\im ka}\qty(1-r_0\ee{2\im ka}+(-r_0\ee{2\im ka})^2+\cdots)\\
&= r_0 -\frac{s_0^2\ee{2\im ka}}{1+r_0\ee{2\im ka}} = \frac{r_0-(1+2r_0)\,\ee{2\im ka}}{1+r_0\ee{2\im ka}}
= \frac{\ee{-\im ka}-(1+\im\theta)\,\ee{\im ka}}{(-1+\im\theta)\,\ee{-\im ka}+\ee{\im ka}}}
与法一结果相同。
\end{enumerate}

\begin{enumerate}[label=\textbf{5.\Alph*}, listparindent=\parindent, leftmargin=-0.5mm]
\setcounter{enumi}{1}
\item \emph{作业5.20题加分小问:计算能量满足$E<E_0$ ($E_0$很大)系统的束缚态能级数量,再从相空间的角度计算束缚态能级数量,二者比较结果。(注:$z$向动能不计入总能量)}

\noindent【解答】

本题讨论二维谐振子模型,能级为
\[E = (n_1+n_2+1)\hbar \omega\]
对$E<E_0$的能级,要求$n_1+n_2+1<\frac{E_0}{\hbar \omega}$,其中$\frac{E_0}{\hbar \omega}\gg 1$。$n_1,\,n_2$取非负整数,这样的能级数量约为$N = \frac{1}{2}\qty(\frac{E_0}{\hbar\omega})^2$个。

再从相空间考虑,设二维谐振子
\[H = \frac{p_x^2}{2m}+\frac{p_y^2}{2m}+\frac{1}{2}m\omega^2 x^2+\frac{1}{2}m\omega^2 y^2,\]
相空间为$(x,\,y,\,p_x,\,p_y)$构成的四维空间。相空间上满足$H<E_0$的点所构成区域(即经典可达到的点构成的区域)应为四维椭球。需要知道半径为$R$的四维球的超体积公式为
\[V = \frac{1}{2}\pi^2R^4,\quad(\text{$N$维球超体积公式为$V = \frac{\pi^\frac{N}{2}}{\Gamma(1+\frac{N}{2})}R^N$})\]
则四维椭球的超体积为
\[V = \frac{1}{2}\pi^2\cdot\sqrt{2mE_0}\cdot \sqrt{2mE_0} \cdot \sqrt{\frac{2E_0}{m\omega^2}} \cdot \sqrt{\frac{2E_0}{m\omega^2}} = \frac{2\pi^2E_0^2}{\omega^2}\]
若量子态个数为$n$,应当有$V \approx nh$,因此
\[n = \frac{2\pi^2E_0^2}{h^2\omega^2 } = \frac{E_0^2}{2\hbar^2\omega^2}\]
与直接求解能级个数的结果相同。

\item \emph{题目:
一电子质量$\mu$,电荷$-e$被限制在$R_1<\rho<R_2$, $0<z<L$的中空圆柱壳内运动。
\begin{enumerate}
    \item 求本征波函数(无需归一化)。证明本征能量为
    \[E_{\ell mn}=\frac{\hbar^2}{2\mu}\qty[k_{mn}^2+\qty(\frac{\ell\pi}{L})^2]\]
    其中$k_{mn}$为以下方程的第$n$个根:
    \[\J_m(k_{mn}R_2)\N_m(k_{mn}R_1)=\J_m(k_{mn}R_1)\N_m(k_{mn}R_2).\]
    \item
    在$0<\rho<R_1$区域内施加$\vec{B}=B\hat{\vec{z}}$磁场,求电子本征能量.
    \item
    证明施加如上磁场后,基态能量不变的条件为
    \[\pi R_1^2 B = \frac{2\pi N \hbar c}{e},\quad (N=0,\,\pm1,\,\pm2,\cdots).\]
\end{enumerate}}

\noindent【解答】

\noindent (a)

容易看出本题选极坐标最为方便。在$R_1<\rho<R_2$区间,薛定谔方程为
\[\hH\psi = -\frac{\hbar^2}{2\mu}\qty[\frac{1}{\rho}\pdv{\rho}\qty(\rho\pdv{\rho})+\frac{1}{\rho^2}\pdv[2]{\varphi}+\pdv[2]{z}]\psi = E\psi\]
分离变量$\psi(\rho,\,\varphi,\,z)=R(\rho)\Phi(\varphi)Z(z)$,代入可得
\[\frac{1}{R(\rho)}\,\frac{1}{\rho}\dv{\rho}\qty(\rho\dv{R(\rho)}{\rho})+\frac{1}{\rho^2}\frac{\Psi''(\varphi)}{\Psi(\varphi)}+\frac{Z''(z)}{Z(z)} + \frac{2\mu E}{\hbar^2}=0\]
波函数边界条件为$Z(0)=0$, $Z(L)=0$, $R(R_1)=0$, $R(R_2)=0$。首先$Z''/Z$一定为常数,为满足$z$的边条件,只能有
\[Z(z)=A\sin(\frac{\ell \pi}{L}z),\quad \ell=1,2,\cdots\]
于是
\[\frac{\rho^2}{R}\qty(\frac{1}{\rho} R'(\rho)+R''(\rho))+\rho^2\qty(\frac{2\mu E}{\hbar^2}-\frac{\ell^2 \pi^2}{L^2})+\frac{\Psi''(\varphi)}{\Psi(\varphi)}=0\]
因此$\Psi''/\Psi$一定为常数,为满足周期性边界条件$\Psi(\varphi)=\Psi(\varphi+2\pi)$,只能有
\[\Psi(\varphi)=B\,\ee{\im m\varphi},\quad m=0,\pm1,\pm2,\cdots\]
于是
\begin{equation}\label{eq:1}
    \frac{1}{\rho}R'+R''+\qty(\frac{2\mu E}{\hbar^2}-\frac{\ell^2\pi^2}{L^2}-\frac{m^2}{\rho^2})R=0
\end{equation}
设$k^2 = \frac{2\mu E}{\hbar^2}-\frac{\ell^2\pi^2}{L^2}>0$ (即要求$z$方向动能不能太高), 令$\xi =k\rho$,方程可化为标准Bessel方程。解得径向波函数
\[R(\rho)=C\J_m(k\rho) + D\N_m(k\rho)\]
需满足边条件
\alg{\begin{cases} C\J_m(kR_1) + D\N_m(kR_1)=0\\
C\J_m(kR_2) + D\N_m(kR_2)=0\end{cases}}
有非零的$C,\,D$解说明两式线性相关,即要求
\[\J_m(kR_1)\N_m(kR_2)-\J_m(kR_2)\N_m(kR_1)=0\]
满足上式的第$n$个$k$记为$k_{mn}$。将三项拼起来可得本征波函数
\alg{\psi(\rho,\,\varphi,\,z)=C'\,\Big[\N_m(k_{mn} R_1)\J_m(k_{mn} \rho)-\J_m(k_{mn} R_1)&\N_m(k_{mn} \rho)\Big]\,\ee{\im m\varphi}\sin(\frac{\ell \pi}{L}z),\\
& \text{$\ell,\,m,\,k_{mn}$取值见上}}
本征能量由$k$表达式解得
\[E_{\ell mn} =\frac{\hbar^2}{2\mu}\qty[k_{mn}^2+\qty(\frac{\ell\pi}{L})^2]\]

\noindent (b)

在$\rho<R_1$区域内施加$z$方向恒磁场,根据$\curl{\vec{A}}=\vec{B}=B\,\vec{e}_z$。在$R_1<\rho<R_2$区域内,由对称性和环路定理$A(\rho)\,2\pi\rho = \pi R_1^2 B$解得
\[\vec{A}(\rho)=\frac{R_1^2B}{2\rho}\,\vec{e}_\varphi,\quad R_1<\rho<R_2\]
(或者,数学上看可以类比电流产生磁场$\curl{\vec{B}}=\mu_0 \vec{j}$,$\vec{A}$场的分布数学上等同与柱形$z$方向的电流在周围形成的磁场$\vec{B}$的分布。)

由于有$\div{\vec{A}}=0$,于是$\hp\vec{\cdot A}-\vec{A\cdot }\hp=-\im\hbar\div{\vec{A}}=0$,故新的薛定谔方程为(注意电荷量$q=-e$)
\[\hat{H}=\frac{1}{2\mu}\qty(\hp+\frac{e\vec{A}(\vec{r})}{c})^2 = \frac{1}{2\mu}\qty(\hp^2+\frac{e^2\vec{A}^2(\vec{r})}{c^2}+2\vec{A}(\vec{r})\vec{\cdot}\hp)\]
第三项为
\alg{\vec{A}(\vec{r})\vec{\cdot}\hp &= (-\im\hbar)\frac{eR_1^2 B}{2c}\qty(-\frac{y}{\rho^2},\,\frac{x}{\rho^2},\,0)\vec{\cdot}\qty(\pdv{x},\,\pdv{y},\,\pdv{z})\\
&= (-\im\hbar)\frac{eR_1^2 B}{2c}\qty(-\frac{y}{\rho^2}\pdv{x} + \frac{x}{\rho^2}\pdv{y})\\
&= (-\im\hbar)\frac{eR_1^2 B}{2c}\frac{1}{\rho^2}\pdv{\varphi}}
因此将第二、第三项代入并将第一项在柱坐标下展开后,发现可以对$\pdv*{\varphi}$配方:
\alg{\hH\psi &= -\frac{\hbar^2}{2\mu}\qty[\frac{1}{\rho}\pdv{\rho}\qty(\rho\pdv{\rho})+\frac{1}{\rho^2}\pdv[2]{\varphi}+\pdv[2]{z} - \frac{e^2R_1^4B^2}{4\hbar^2 c^2}\frac{1}{\rho^2}+
\frac{\im eR_1^2 B}{\hbar c}\frac{1}{\rho^2}\pdv{\varphi}]\psi \\
&= -\frac{\hbar^2}{2\mu}\qty[\frac{1}{\rho}\pdv{\rho}\qty(\rho\pdv{\rho})+\frac{1}{\rho^2}\qty(\pdv{\varphi}+\frac{\im eR_1^2 B}{2\hbar c} )^2+\pdv[2]{z}]\psi}
令$\beta = \frac{eR_1^2 B}{2\hbar c}$进行无量纲化。仿照第(a)问分离变量的步骤,$Z(z)$函数没有变,$\Phi(\varphi)$的方程变为
\[\frac{1}{\Phi}\qty(\pdv{\varphi}+\im \beta)^2\Phi = \text{常数}=\lambda\]
即二阶常系数微分方程
\[\Phi'' + 2\im\beta\Phi'-(\beta^2+\lambda) \Phi =0\]
考虑试探解$\Phi = \ee{\im m \varphi}$,要求周期边界条件需要$m=0,\pm1,\pm2,\cdots$。代入有
\[-m^2-2\beta m-\beta^2-\lambda = -(m+\beta)^2-\lambda = 0\]
因此$\lambda = -(m+\beta)^2$。这与(a)问的$\lambda=-m^2$有所不同,会影响径向波函数的参数,但角向波函数仍为$\Phi(\varphi)=B\ee{\im m \varphi}$。

令$m' = m+\beta$,继续求解径向波函数时,只需用$m'$替换第(a)中的$m$即可,因此径向波函数通解是
\[R(\rho) = C\J_{m'}(k\rho)+D\N_{m'}(k\rho)\]
最终波函数解为
\alg{\psi(\rho,\,\varphi,\,z)=C'\,\Big[\N_{m'}(k_{m'n} R_1)\J_{m'}(k_{m'n} \rho)-\J_{m'}(k_{m'n} R_1)&\N_{m'}(k_{m'n} \rho)\Big]\,\ee{\im m\varphi}\sin(\frac{\ell \pi}{L}z)}
$k_{m'n}$是以下方程的第$n$个根
\[\J_{m'}(kR_1)\N_{m'}(kR_2)-\J_{m'}(kR_2)\N_{m'}(kR_1)=0\]
可见只有径向波函数发生了变化。相应的本征能量是
\[E_{\ell mn} =\frac{\hbar^2}{2\mu}\qty[k_{m'n}^2+\qty(\frac{\ell\pi}{L})^2],\quad \text{$m' = m+\frac{eR_1^2 B}{2\hbar c},\quad k_{m'n}$由上式定义.}\]

\noindent(c) 

一个直观的感觉是:当$\beta$为整数时,上述$m'$依旧是整数,只相当于对原整数格点做了平移后再次重合,能级未发生变化,基态能级当然也不会变化。但我们需要严谨说明之。

我们作如下论证:$\abs{m}$越小,超越方程的解$k_{mn}$越小。我们可以巧用HF定理证明之。根据(\ref{eq:1})式的特征构造一个径向波函数的哈密顿量:
\[\hat{H}_0=\begin{dcases}-\frac{\hbar^2}{2\mu}\qty(\dv[2]{\rho}+\frac{1}{\rho}\dv{\rho}-\frac{\ell^2\pi^2}{L^2}-\frac{m^2}{\rho^2}),&R_1<\rho<R_2\\
\infty,&\text{其他}
\end{dcases}\]
则(a)问中径向波函数解$R_{mn}(\rho)$满足$\hat{H}_0R_{mn}(\rho)=E_{\ell mn}R_{mn}(\rho)$。
请注意这一哈密顿量并不具有实际意义,它甚至不是厄米的。但这不影响HF定理的使用。将$m$升级为连续参量,考虑$\hat{H}_0$,$E_{\ell mn}$, $R_{mn}(\rho)$随$m$的变化,则有
\[\mel**{R_{mn}}{\pdv{\hat{H}_0}{m}}{R_{mn}}=\mel**{R_{mn}}{\frac{\hbar^2}{2\mu}\frac{2m}{\rho^2}}{R_{mn}} = \pdv{E_{\ell mn}}{m}\]
可以看出当$m>0$时,上式为正,$E_{\ell mn}$随$m$的增加而增加;$m<0$时,上式为负,从而$E_{\ell mn}$随$m$的减小也增加。于是$m=0$时能量是最小的。对应到(a)问的情形,基态应该对应$\ell=1$,$m=0$,$n=1$ (第1个解)。对(b)问,应有$\ell=1$,$m'=m+\beta$绝对值最小,$n=1$时达到基态。所以只有$\beta=\frac{eR_1^2B}{2\hbar c}$为整数时$\abs{m'}$最小值才为0,基态能量与不加磁场的情形相同。于是
\[\pi R_1^2B=\frac{2\pi N\hbar c}{e},\quad N=0,\pm1,\pm2,\cdots\]
磁通量出现“量子化”的现象。

\noindent{\color{red}\textbf{【讨论】}}

根据电磁场规范变换的理论,若对磁矢势场做变换$\vec{A}(\vec{r})\rightarrow \vec{A'}(\vec{r})=\vec{A}(\vec{r})+\grad{\chi}(\vec{r})$,则只需对波函数作相位变换
\[\psi(\vec{r})\rightarrow \psi'(\vec{r})=\ee{-\im e\chi(\vec{r})/\hbar c}\,\psi(\vec{r})\]
即可得到变换后的矢势下的波函数解。注意这里电荷$q=-e$。

(b)问相比(a)问在$R_1<\rho<R_2$多了矢势场$\vec{A}(\rho)=\frac{R_1^2B}{2\rho}\vec{e}_\varphi$。因此可以构造函数
\[\chi(\rho,\varphi,z)=\frac{R_1^2B}{2}\varphi=\frac{\hbar c \beta}{e}\varphi,\quad \varphi\in(0,\,2\pi-\epsilon)\]
它在$0<\varphi<2\pi-\epsilon$区间内使$\vec{A}=\grad{\chi}$成立。当然不可能在全空间找到这样的$\chi$,因为$\vec{A}$在$\varphi\in(0,2\pi]$内将是有旋的,而任意梯度场都是无旋的。不过这至少可让我们在$0<\varphi<2\pi-\epsilon$区间内,对(a)求得的波函数作变换
\[\psi(\rho,\varphi,z)\rightarrow \psi'(\rho,\varphi,z)=\ee{-\im e\chi/\hbar c}\,\psi(\rho,\varphi,z)
= \ee{-\im \beta}\,\psi(\rho,\varphi,z)\]
它在$0<\varphi<2\pi-\epsilon$区间内是满足(b)问的薛定谔方程的。
显然它只变动了(a)的角向波函数
\[\Phi(\varphi)\rightarrow \Phi'(\varphi) =\ee{-\im\beta}\,\Phi(\varphi)= B\,\ee{\im(m-\beta)}\]
我们不能轻易将$\varphi\in(0,\,2\pi-\epsilon)$扩展至$(0,2\pi]$范围,因为角向的周期性边条件将不再满足。但我们可大胆地将$m$改成$m'$使得$m'-\beta$为整数,这样角向波函数即可扩展至$(0,2\pi]$范围了,而$m$的变动将会导致径向波函数的参数发生变化,这会导致本征能量发生改变。但改变后的波函数将会在全空间满足薛定谔方程,相当于我们直接猜出了(b)的答案。

\end{enumerate}