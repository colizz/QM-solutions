{\color{red}下为根据刘老师量子力学讲义第一章作业制作的习题解答。请大家尤其重点看一下1.4, 1.8, 1.13, 1.24, 1.31, 1.32, 1.38, 1.40, 1.41题及红字标注的部分,这是大家出错率较高的及我们认为有必要提醒大家的重点题目。周五习题课上我们会讲解部分习题。}

常用物理量:$ \sigma=5.6704\ten{-8}\;\mathrm{J/(K^4\cdot s\cdot m^2)} $,
$ h = 6.6261\ten{-34}\;\mathrm{kg\cdot m^2/s} $,
$ hc=1240\;\mathrm{eV\cdot nm} $

\begin{enumerate}[label=1.\arabic*]

\item 
黑色平板吸收与辐射平衡
$ E_{in}=E_{a}= 1350\; \mathrm{J/(s\cdot m^2)}$。
将平板视为黑体,由Stefan-Boltzmann定律有
\[E_{in}=2\sigma T^4.\]
{\color{red}注意到因平板有两面,所以要加一个因子\,2}。解得$ T=330\;\mathrm{K} = 57\ssd $。

\item
地球上$ S_0=1\cm^2 $的区域于$T_0=60\s$接受辐射能量$E_0=8.11\;\mathrm{J}$ ,
太阳光的能流密度为\[E=\frac{4\_pi L^2}{4\pi(D/2)^2}\cdot \frac{E_0}{S_0T_0} \]
太阳温度为5778 K,则由Stefan-Boltzmann定律可求得
\[\sigma = \frac{4\pi L^2}{4\pi(D/2)^2}\cdot \frac{E_0}{S_0T_0}\cdot \frac{1}{T^4}=5.65\ten{-8}\;\mathrm{J/(K^4\cdot s\cdot m^2)}.\]

\item
波长在550 nm到551 nm内可认为辐射功率随波长变化很小,因此
\[P\approx r_B(\lambda,\;T)\Delta \lambda\cdot S.\]
由普朗克公式
\[r_B(\lambda,\;T) = \frac{2\pi hc^2}{\lambda^5}\frac{1}{e^{\frac{hc}{\lambda k_B T}-1}}.\]
并代入$\lambda=1.0\nm$,$S=\pi(D/2)^2$,$T=5973\kai$得到$P=7.40\ten{-4}\;\mathrm{W}$。

单位时间内光子发射数目为
\[N = \frac{P}{hc/\lambda} = 2.05\ten{15}\;\text{个/s}.\]

\item
(1) 求辐射场温度。{\color{red}对光子场,能量密度$U$与能流密度$J$满足$J=\frac{1}{4}cU$}。现简要推导如下(详见热力学统计物理教材):

考虑辐射场中射向$\dd{A}$小面元的辐射能量。在$\dd{\Omega}$立体角内单位时间射向面元的能量为$cU\cos\theta\dd{A}\frac{\dd{\Omega}}{4\pi}$,$\theta$为辐射方向与面元法线夹角。对立体角在面元一侧的半空间积分得
\[J\dd{A} = \int\frac{cU\dd{A}\cos\theta}{4\pi}\dd{\varOmega} = \frac{cU\dd{A}}{4\pi}\int_0^{\frac{\pi}{2}}\cos\theta\sin\theta\dd{\theta}\int_0^{2\pi}\dd{\varphi}=\frac{cU\dd{A}}{4}.\]
考虑在该辐射场中的黑体,达到热平衡时温度为$T$,此时应有辐射能量等于吸收的场能,故
\[J=\frac{1}{4}cU=\sigma T^4.\]
得到$T = 2.82\kai$,此即微波背景辐射的温度。

(2) 求光子数密度。辐射场的辐射能流$E(T)=\int_0^\infty r_B(\lambda,\;T )\dd{\lambda}$,其中$r_B(\lambda,\;T )$形式由普朗克公式给出。根据辐射能量与能流的关系,可知$\lambda$到$\lambda+\dd{\lambda}$区间内光子的能量密度为$u(\lambda,\;T)\dd{\lambda}= \frac 4c r_B(\lambda,\;T)\dd{\lambda}$,进一步可知光子数密度为
$n(\lambda,\;T)\dd{\lambda} = \frac{1}{hc/\lambda}u(\lambda,\;T)\dd{\lambda}$。
因此总光子数密度为
\[N(T)=\int_0^{\infty}n(\lambda,\;T)\dd{\lambda}=\int_0^{\infty}\frac{4\lambda}{hc^2}r_B(\lambda,\,T)\dd{\lambda} = \int_0^{\infty}\frac{8\pi}{\lambda^4}\frac{1}{e^{\frac{hc}{\lambda k_B T}}-1}\dd{\lambda}\]
换元$x=\frac{hc}{\lambda k_B T}$并作数值积分有
\[N(T)=8\pi \left(\frac{k_B T}{hc}\right)^3\int_0^\infty \frac{x^2}{e^x -1}\dd{x} = 2.404\times8\pi\left(\frac{k_B T}{hc}\right)^3,\]
代入温度$T$得到$N=4.5\ten{8}\;\mathrm{m^{-3}}$.

(注:本题可直接由统计物理中的光子气体分布推出。由玻色-爱因斯坦分布可导出与上面完全一致的形式。详见统计学课本。)

(3) 平均光子能量
\[\bar{\epsilon}=\frac{\int_0^{\infty}u(\lambda,\;T)\dd{\lambda}}{\int_0^{\infty}\frac{u(\lambda,\;T)}{hc/\lambda}\dd{\lambda}}=\frac{8\pi\frac{k_B^4 T^4}{h^3 c^3}\frac{\pi^4}{15}}{8\pi\left(\frac{k_B T}{hc}\right)^3\times 2.404} = \frac{\pi^4}{15\times2.404}k_B T,\]
代入温度得到$\bar\epsilon = 1.1\ten{-22}\;\mathrm{J}.$

(4) 由维恩位移定律可知最大亮度对应波长为$\lambda_M = b/T= 1.0\mm$。


\item
蓝光波长$400\sim 450 \nm$。
取$\lambda_M=400\nm$作为辐射峰值所在波长计算,由维恩位移定律$\lambda_M T=b=2.898\ten{-3}\;\mathrm{m\cdot K}$,得到$T=7.2\ten{3}\kai=7.0\ten{3}\ssd$

\item
考虑温度为$T = 1.0\ten{7}$,由维恩位移定律$\lambda_M T=b$
可得$\lambda_M = 0.29\nm$,
光子能量为$E=hc/\lambda_M=4.3\ten{3}\eV=6.9\ten{-16}\;\mathrm{J}$。

\item
将人视为高为$h=1.8$ m的圆柱形,截面周长以腰围估算,取
$L=1.0$ m。
故辐射面积为$S=hL$。
人体体表温度$T = 33.5\ssd$,由Stefan-Boltzmann定律求得
辐射功率为$P=S\sigma T^4= 9.0\ten{2}$ W。

\item
可以将能流密度各自在频率$\lambda$与波长$\nu$上写作积分形式:
\[E(T) = \int_0^{\infty}r_B'(\nu,\,T)\dd{\nu} = \int_0^{\infty}r_B(\lambda,\,T)\dd{\lambda}. \]
我们实际上要求的是$r_B'(\nu,\,T)$取极值时所对应的$\nu_m$的表达式。根据
\eqa{\lvert r_B'(\nu,\,T)\mathrm{d}\nu\rvert=\lvert r_B(\lambda,\,T)\mathrm{d}\lambda\rvert}
辐射强度随频率变化的关系为
\eqa{r_B'(\nu,\,T)=r_B(\lambda,\,T)\Big\lvert\frac{\mathrm{d}\lambda}{\mathrm{d}\nu}\Big\rvert}
波长和频率的关系为$\,\lambda=c/\nu$,故
\eqa{\Abs{\drv{\lambda}{\nu}}=\frac{c}{\nu^2}}
所以
\alg{r_B'(\nu,\,T)=&r_B(\lambda,\,T)\cdot\frac{c}{\nu^2}\\
=&\frac{2\pi h}{c^2}\frac{\nu^3}{\exp(\frac{h\nu}{\kb T})-1}}
我们需要求$\,r_B'(\nu,\,T)\,$的极大值点$\,\nu_m$,令$\,\prv{r_B(\nu,\,T)}{\nu}=0$,得到方程
\eqa{x\epow{x}-3\epow{x}+3=0}
其中$\,x=\frac{h\nu_m}{\kb T}$。数值解为
\eqa{x=2.82}
则维恩位移定律的频率形式为
\eqa{\nu_m=2.82\frac{\kb T}{h}}

{\color{red}(注:绝大多数同学使用了$\nu_m = c/\lambda_m$,错误地直接由维恩位移定律的波长形式得到频率形式。事实上$\nu_m$与$\lambda_m$对应的是完全不同的函数取极值下的频率与波长,该极值点也并不在同一个波长/频率区间内取得。另外,上面过程中的$r_B'(\nu,\,T)$与$r_B(\lambda,\,T)$也是完全不同的函数,并不是简单地替换了自变量。请一定注意。)}

\item
(1)由维恩位移定律,最敏感波长应为$\lambda_M= b/T_0 = 9.27\mum$.

(2)总频谱亮度可由Stefan-Boltzmann定律求得为$E=\sigma T^4$,故高出的百分比
\[\frac{E'-E_0}{E_0} = \frac{T'^4-T_0^4}{T_0^4} = 8.1\%\]

\item
单个光子能量为$\epsilon = \frac{hc}{\lambda}$。
设每秒进入眼睛光子数为$N$,则光强$\frac{N\epsilon}{S}>10^{-10}\;\mathrm{W/m^2}$。
求得$N>3.67\ten{3}\;\mathrm{s^{-1}}$.

\item
单个光子能量为$\epsilon = \frac{hc}{\lambda}$,单位时间粒子数为$N$。功率为$P = N\epsilon = 3.61\ten{-17}\;\mathrm{W} = 225\;\mathrm{eV/s}$。

\item
$\Delta t$时间内,照到镜面单位面积上的光子数为$N\Delta t$,则照到单位面积上光子的原动量为
\[p = N\Delta t\cdot \frac{h\nu}{c} = \frac{Nh\nu\Delta t}{c}.\]
动量改变量
$\Delta p = p-(-p) = 2p $.
根据牛顿定律,$\Delta p = P_\text{压强}\Delta t$,则
\[P_\text{压强} = \frac{2Nh\nu}{c}\]
如果考虑光子不是正入射,设入射角为$\,\theta$,显然单个光子动量变化为$\,\Delta p_1=2\frac{h\nu}{c}\cos\theta$,单位面积上入射的光子数为$\,N\Delta t\cos\theta$,故光子总动量的改变为
\eqa{\Delta p=\Delta p_1N\Delta t\cos\theta}
根据牛顿定律,压强为
\eqa{P_{压强}=\frac{2Nh\nu}{c}\cos^2\theta}

\item
考虑截面为$S$,长度为$L$的空间。单个光子能量$\epsilon = h\nu$,则单位时间内照在单位面积上的辐射能量
\[E = \frac{N\epsilon SL}{S\cdot l/c} = N\epsilon c.\]
已知$E=1\;\mathrm{J/(cm^2\cdot s)}$。

(1) 对于$\nu=1\;\mathrm{MHz}$的无线电波,计算得粒子数密度为
$N = 5.03\ten{22}\;\mathrm{m^{-3}}$。

(2) 对于$E=h\nu=1\MeV$的光子,计算得粒子数密度为
$N = 2.08\ten{7}\;\mathrm{m^{-3}}$。

{\color{red}(注:本题中不能单纯地仿效1.4题得出$E = \frac{N\epsilon c}{4}$。这是因为题目中强调了考察对象是“一束”电磁波,这说明所有光子都是同方向的。这与1.4题中“各向同性”的光子场不同。)}

\item%14
假设自由电子可以发生光电效应。假定一个光子(动量为$\,\mathbf{p}_1$)与一个自由电子(动量为$\,\mathbf{p}_2$)碰撞,电子吸收光子,动量变为$\,\mathbf{p}_3$,$\mathbf{p}_1\,$与$\,\mathbf{p}_3\,$的夹角为$\,\theta$,则有动量守恒和能量守恒两个方程:
\alg{p_3^2+p_1^2-2p_1p_3\cos\theta=&p_2^2\\
p_1c+\sqrt{p_2^2c^2+E_0^2}=&\sqrt{p_3^2c^2+E_0^2}}
其中$\,E_0\,$为电子的静质量能。整理可得
\eqa{p_3^2c^2\sin^2\theta=-E_0^2}
此式左边是非负数而右边是负数,显然不可能成立。故自由电子不可能完全吸收一个光子而产生光电效应。

【另解】若一个光子与电子发生作用,末态只有一个电子,则我们可以考虑末态电子的静止系。根据能量守恒
\[E_{e\text{初}} + E_\gamma = E_{e\text{末}} = m_e c^2.\]
而显然初态电子具有非零动量,即$E_{e\text{初}}>m_e c^2$,故上式不可能成立。

\item
红限波长$\lambda_0 = 600\nm$,则逸出功为$W=\frac{hc}{\lambda_0}$。
设某单色光波长$\lambda$,由光电效应原理
\[\frac{hc}{\lambda}-W = eU_0.\]
解得
\[\lambda=\frac{hc}{eU_0+hc/\lambda_0} = 272\nm.\]

\item
根据$\frac{hc}{\lambda}-W = eU_0$,变形为
\[U_0 = \frac{hc}{e}\cdot \frac{1}{\lambda}-\frac{W}{e}.\]
因此可知$U_0$与$\frac{1}{\lambda}$呈线性关系。作最小二乘法拟合可得到斜率$\frac{hc}{e} = 1217\;\mathrm{V\cdot nm}$。由此求得普朗克常量$h=6.50\ten{-34}\;\mathrm{J\cdot m}$。

\item
截止波长$\lambda_0$满足$\frac{hc}{\lambda_0}=W$,解得$\lambda_0 = 539\nm$。若采用$\lambda=680\nm$的波长,则$\frac{hc}{\lambda}-W<0$,故不能发生光电效应。

\item
考虑红光波长$\lambda=760\nm$,其单个光子能量为$\epsilon = \frac{hc}{\lambda}=2.62\ten{-19}\;\mathrm{J}>10^{-19}\;\mathrm{J}$.
故可以利用溴化银底版在红光下进行拍照。

\item
根据能守恒原理,反冲电子的动能来源于光子能量的损失,即$E_R = E_{\gamma,I}-E_{\gamma,R}$。
考虑到康普顿散射效应中,出射光子波长与入射光子波长满足
\[\lambda'-\lambda = \frac{h}{m_e c}(1-\cos\theta).\]
而入射光子能量为$E_{\gamma,I}=\frac{hc}{\lambda}$已知,则可求得出射电子动能为
\[E_R = \frac{hc}{\lambda}-\frac{hc}{\lambda'} = E_{\gamma,I} - \frac{E_{\gamma,I}}{1+\frac{E_{\gamma,I}}{m_e c^2}(1-\cos\theta)}.\]

\item
设出射电子动量大小为$p$,入射、出射光子频率为$\nu,\,\nu'$。根据过程前后竖直(垂直于入射光子速度方向)、水平动量守恒:
\[
\begin{cases}
p\sin\phi = \frac{h\nu'}{c}\sin\theta \\
p\cos\phi = \frac{h\nu}{c}-\frac{h\nu'}{c}\cos\theta
\end{cases}
\]
二式作比得
\[\cot\phi = \frac{\nu-\nu'\cos\theta}{\nu'\sin\theta}.\]
由于$\frac{1-\cos\theta}{\sin\theta} = \tan\frac{\theta}{2}$,同时根据康普顿效应的波长差公式可得频率关系式
\[\frac{\nu-\nu'}{\nu\nu'} = \frac{h}{m_e c^2}(1-\cos\theta),\]
从而
\[\cot\phi = \nu'\frac{1-\cos\theta}{\sin\theta}+\frac{\nu-\nu'}{\nu'\sin\theta} = \nu'\tan\frac{\theta}{2}+\frac{\nu\nu'(1-\cos\theta)\frac{h}{m_e c^2}}{\nu'\sin\theta} = (1+\frac{h\nu}{m_e c^2})\tan\frac{\theta}{2}.\]
即为所求。

\item
散射前、后光子能量为$E_\nu,\,E_\nu'$,则根据康普顿散射公式:
\[\Delta \lambda = \frac{hc}{E_\gamma'}-\frac{hc}{E_\gamma} = \frac{hc}{m_e c}(1-\cos\theta),\]
求解得$\theta=41.9\du$。

\item
已知光子散射角$\theta = 90\du$。由康普顿散射公式可求得
\[\lambda' = \lambda+\frac{h}{m_e c}(1+\cos\theta) = \frac{hc}{E_\nu}+\frac{h}{m_e c} = 2.53\;\mathrm{pm}.\]

\item
设题述$5.00\MeV$为入射电子的动能,记为$E_k$,出射光子能量记为$E_{\gamma_1},\,E_{\gamma_2}$,$E_{\gamma_1}$朝向入射电子方向,则有动量、能量守恒关系式:
\[
\begin{cases}
p = \frac{1}{c}\sqrt{(m_e c^2 + E_k^2)-m_e^2 c^4} = \frac{1}{c}(E_{\gamma_1}-E_{\gamma_2})
\\
2m_e c^2 + E_k = E_{\gamma_1}+E_{\gamma_2}
\end{cases}
\]
两式相加、相减得$E_{\gamma_{1,2}} = m_e c^2+\frac{E_k}{2} \pm \frac{1}{2}\sqrt{E_k(E_k+2m_e c^2)}$,计算得$E_{\gamma_1}=5.76\MeV$,$E_{\gamma_2}=0.268\MeV$。

\item
{\color{red}首先请注意,单个光子不可能湮灭为一对正负电子。}请同学们思考其中的原因(在近代物理知识范围内)。于是我们猜测题目本意为散射光子能量应至少大于两倍电子静质量,即$E_\gamma'>2m_e c^2$。
根据康普顿散射公式,有
\[\frac{h}{m_e c}(1-\cos\theta) = \frac{hc}{E_\gamma'}-\frac{hc}{E_\gamma} < \frac{hc}{E_\gamma'}<\frac{h}{2m_e c},\]
从而$\cos\theta>\frac{1}{2}$,即$\theta<60\du$。

\item
光子垂直出射,$\theta=90\du$,根据康普顿散射公式,偏移量
\[\Delta\lambda = \frac{h}{m_e c}(1-\cos\theta) = 2.42\;\mathrm{pm}.\]
出射光子能量为$E_\gamma' = \frac{hc}{\lambda+\Delta\lambda}$,已知$\gamma$射线波长为$0.00188\nm$,则入射光损失的能量占总能量比为
\[\frac{E_\gamma'-E_\gamma}{E_\gamma} - \frac{\frac{1}{\lambda}-\frac{1}{\lambda+\Delta\lambda}}{\frac{1}{\lambda}} = \frac{\Delta\lambda}{\lambda+\Delta\lambda} = 56.3\%.\]
反冲动能等于光子能量损失,即
\[E_R = E_\gamma-E_\gamma' = \frac{hc}{\lambda}-\frac{hc}{\lambda+\Delta\lambda} = 5.96\ten{-14}\J = 0.372\MeV.\]

\item
卢瑟福散射公式是对单一点状原子核散射粒子所形成的分布的描述。引入瞄准距离$b$为入射粒子到原子核的垂直距离,则$\theta \rightarrow 0$实际对应着$b \rightarrow \infty$,这表示入射在$b>b_0$无穷大横截面上的粒子理论上都可以被散射进$\theta < \theta_0$的区域内,这自然造成了$\dv{\sigma(\theta)}{\Omega}$在$\theta \rightarrow 0$的发散。然而实际中,散射粒子不仅受到一个原子核的散射作用,对于以某个原子核为参考的$b$较大的入射粒子定会受到其他原子核的影响。故处理实际问题中应对每个原子核加以作用域的限制,超出该作用域范围的粒子将由其它原子核主导其散射行为。

\item
(略)

\item
在质子最接近Au原子时,动能完全转化为电势能,则有
\[\frac{1}{4\pi\epsilon_0}\frac{Ze^2}{r} = E_k.\]
Au元素$Z=79$,求解得$r=56.8\fm$。考虑到质子与Au核的半径,我们有$r>r_p+r_\mathrm{Au} = 8.9\fm$,因此质子与Au核不能接触以发生核反应。

\item
产生该结果的主要原因是:质子实际具有质量,因此质子-电子体系应当作为两体模型计算。我们下面求解考虑质子质量的情况下对里德堡常量的修正。

设质子质量$M$,电子质量$m$,二者绕共同质心作圆周运动,距离为$r$。由牛顿第二定律
\[\frac{1}{4\pi\epsilon_0}\frac{e^2}{r^2} = \frac{mv^2}{\frac{rM}{M+m}}.\]
角动量的量子化条件为
\[L = mvr\frac{M}{M+m}+M\frac{vm}{M}\frac{rm}{M+m} = mvr = n\hbar.\]
因此$r$只能取分立值:
\[\frac{1}{r} = \frac{e^2 m}{4\pi\epsilon^2\hbar^2}\frac{M}{M+m}\]
总能量为
\[E = \frac{1}{2}mv^2+\frac{1}{2}M\frac{v^2m^2}{M^2}-\frac{1}{4\pi\epsilon_0}\frac{e^2}{r}, \]
化简为
\[E = -\frac{1}{2}\frac{1}{4\pi\epsilon_0}\frac{e^2}{r}.\]
代入分立取值的$r$,对比玻尔理论可知里德堡常量的修正应为
\[R' = R\frac{M}{M+m}.\]
考虑到质子与电子质量比$M/m=1836$,可以计算出修正后的里德堡常量与观测值在较高精度内一致。

(注:1.40题所讨论的相对论效应也会对里德伯常量产生修正,参考1.40题结论可知修正量$\frac{\Delta R}{R}\sim \alpha^2 
\approx\frac{1}{137^2}<\frac{1}{1836}$,小于因质子质量带来的修正,故相对论效应并不是主要影响因素。)

\item
考虑到氢原子由基态向上激发,由于
\[E_1(\frac{1}{1^2}-\frac{1}{3^2}) < 12.5\eV < E_1(\frac{1}{1^2}-\frac{1}{4^2}),\]
可知$12.5\eV$的电子可将其激发至$n=2$或$n=3$态。因此受激发的氢原子向低能态跃迁可发出波长为:
\alg{\lambda_{32} &= \frac{hc}{E_1}(\frac{1}{2^2} - \frac{1}{3^2})^{-1}=656.5\nm\\
\lambda_{31} &= \frac{hc}{E_1}(\frac{1}{1^2} - \frac{1}{3^2})^{-1}=102.6\nm\\
\lambda_{21} &= \frac{hc}{E_1}(\frac{1}{1^2} - \frac{1}{2^2})^{-1}=121.6\nm
}

\item
假设电子做半径为$\,r\,$的圆周运动。电子的经典的运动方程为
\eqa{\frac{mv^2}{r}=\frac{1}{4\pi\varepsilon_0r^2}\frac{Ze^2r^3}{R^3}}
由此可得角动量
\eqa{L=mvr=\sqrt{\frac{Ze^2m}{4\pi\varepsilon_0R^3}}r^2}
玻尔理论方法令角动量量子化,有
\eqa{L=\sqrt{\frac{Ze^2m}{4\pi\varepsilon_0R^3}}r^2=n\hbar}
可得
\eqa{r^2=n\hbar\sqrt{\frac{4\pi\varepsilon_0R^3}{Ze^2m}}}
电子受力与距离成正比,其圆周运动可视为两个互相垂直的简谐运动的叠加,其能级应为
\eqa{E_n=2\cdot\frac{1}{2}\frac{Ze^2}{4\pi\varepsilon_0R^3}r^2=n\hbar\sqrt{\frac{Ze^2}{4\pi\varepsilon_0R^3m}}}
由于电子在球内运动,$r^2<R^2$,我们可以确定$\,n\,$的范围:
\eqa{n<\frac{1}{\hbar}\sqrt{\frac{Ze^2mR}{4\pi\varepsilon_0}}}

\item
设$\,\beta=v/c$,由狭义相对论的速度变换可得两个氢原子的相对速度为
\eqa{v_r=\frac{v-(-v)}{1-v(-v)/c^2}=\frac{2\beta c}{1+\beta^2}}
在发出光子的氢原子的参考系中,光子频率为
\eqa{\nu=cR\Brak{1-\frac{1}{2^2}}=\frac{3}{4}cR}
由于多普勒效应,在接收光子的氢原子参考系中,光子频率为
\eqa{\nu'=\sqrt{\frac{1+v_r/c}{1-v_r/c}}\nu=\frac{1+\beta}{1-\beta}\nu}
另一方面,氢原子接收光子后跃迁到第二激发态,说明光子的频率为
\eqa{\nu'=cR\Brak{1-\frac{1}{3^2}}=\frac{8}{9}cR=\frac{32}{27}\nu}
可见
\eqa{\frac{1+\beta}{1-\beta}=\frac{32}{27}}
解之得
\eqa{\beta=0.0862}
即速率$\,v=0.0862c$。\\
由于题目表述不清,也可将$\,v\,$理解为相对速度,此时有
\eqa{\nu'=\sqrt{\frac{1+\beta}{1-\beta}}\nu}
解得
\eqa{\beta=0.168}

\item
首先估算可见光区光子能量的范围,长波端和短波端分别约为
\alg{E_\mathrm{l}=&\frac{hc}{760\,\mathrm{nm}}=1.63\,\mathrm{eV}\\
E_\mathrm{s}=&\frac{hc}{380\,\mathrm{nm}}=3.26\,\mathrm{eV}}
莱曼系光子最低能量为
\eqa{E_\mathrm{ll}^\mathrm{L}=hcR\Brak{\frac{1}{1^2}-\frac{1}{2^2}}=10.2\,\mathrm{eV}>E_\mathrm{s}}
帕邢系光子最高能量为
\eqa{E_\mathrm{sl}^\mathrm{P}=hcR\Brak{\frac{1}{3^2}-\frac{1}{\infty^2}}=1.51\,\mathrm{eV}<E_\mathrm{l}}
可见氢原子光谱中位于可见光区的谱线均出自巴尔末系,$1/\lambda=R\Brak{\frac{1}{2^2}-\frac{1}{m^2}}$,计算出各可见光谱线波长如下表所示
\begin{table}[h]
    \centering
    \begin{tabular}{c|c}
    \hline
        \rule{0pt}{13pt}$m$ & $\lambda/\mathrm{nm}$\\
        \hline
        \rule{0pt}{13pt}3 & 656.3 \\
        \rule{0pt}{13pt}4 & 486.1 \\
        \rule{0pt}{13pt}5 & 434.1 \\
        \rule{0pt}{13pt}6 & 410.2 \\
        \rule{0pt}{13pt}7 & 397.0 \\
        \rule{0pt}{13pt}8 & 388.9 \\
    \hline
    \end{tabular}
    \caption{题\,1.33\,表}
\end{table}

\item
各谱系的长波极限波长为$\,n+1\,$态跃迁到$\,n\,$态对应的波长,各谱系的短波极限波长为$\,\infty\,$态跃迁到$\,n\,$态的波长。莱曼系对应$\,n=1$,巴尔末系对应$\,n=2$,帕邢系为$\,n=3$。计算出结果如下表所示:
\begin{table}[h]
    \centering
    \begin{tabular}{c|c|c}
    \hline
        \rule{0pt}{13pt}谱线系 & $\lambda_\mathrm{ll}/\mathrm{nm}$ & $\lambda_\mathrm{sl}/\mathrm{nm}$ \\
        \hline
        \rule{0pt}{13pt}莱曼系 & 121.6 & 91.15 \\
        \rule{0pt}{13pt}巴尔末系 & 656.3 & 364.6 \\
        \rule{0pt}{13pt}帕邢系 & 1876 & 820.4 \\
    \hline
    \end{tabular}
    \caption{题\,1.34\,表}
\end{table}

\item
氢原子光谱的巴尔末系的最短波长对应氢原子从$\,n=\infty\,$向$\,n=2\,$态跃迁,故
\eqa{\frac{hc}{\lambda}=E_1\Brak{\frac{1}{2^2}-\frac{1}{\infty^2}}}
解得电离能$\,E_1=13.6\,\mathrm{eV}$

\item
发出的光子能量为
\eqa{\frac{hc}{\lambda}=2.54\,\mathrm{eV}}
故初态的激发能为
\eqa{E_i=2.54\,\mathrm{eV}+10.19\,\mathrm{eV}=12.73\,\mathrm{eV}}
结合能为
\eqa{E_b=E_1-E_i=0.87\,\mathrm{eV}}

\item
光子的动量为
\eqa{p=hR\Brak{1-\frac{1}{2^2}}=10.2\,\mathrm{eV}/c}
本题能量远小于氢原子静质量能,故可作经典近似$\,p=m_0v$,则反冲速度为
\eqa{v=\frac{p}{m_0}=3.26\,\mathrm{m/s}}
反冲能为
\eqa{E_k=\frac{1}{2}m_0v^2=55.4\,\mathrm{neV}}
故反冲能与光子能量比值为
\eqa{\frac{E_k}{E_\gamma}=\frac{E_k}{pc}=5.44\et{-9}}

\item
$\mathrm{H}_\alpha\,$线是氢原子由$\,n=3\,$态跃迁到$\,n=2\,$态发出的,则入射电子将氢原子由基态激发到了$\,n=3\,$态,入射电子的最小动能应为氢原子基态与$\,n=3\,$激发态的能量差
\eqa{E_k\leq hcR\Brak{1-\frac{1}{3^2}}=12.1\,\mathrm{eV}}

\item
光子能量在几个$\,\mathrm{eV}\,$量级,远小于质子的静质量能,故可进行经典近似。设质子的质量为$\,m$,初速度为$\,v$,其碰撞损失的动能为质心系中的动能
\eqa{\Delta E=2\cdot\frac{1}{2}m\Brak{\frac{v}{2}}^2=\frac{1}{4}mv^2}
这里忽略了电子的质量。这些能量至少要使氢原子跃迁到第一激发态,即
\eqa{\Delta E=\frac{1}{4}mv^2\geq hcR\Brak{1-\frac{1}{2^2}}=10.2\,\mathrm{eV}}
代入数据可解出
\eqa{v\geq62.5\mathrm{km/s}}

\item 
考虑相对论效应,动量
\[p = \frac{m_0v}{\sqrt{1-v^2/c^2}}.\]
因此角动量量子化条件要求
\[L= \frac{m_0vr}{\sqrt{1-v^2/c^2} }= n\hbar.\]
相对论情形下,电磁力
\[F = \frac{1}{4\pi\epsilon_0}\frac{Ze^2}{r^2} = \abs{\dv{\mathbf{p}}{t}} = \frac{m_0v}{\sqrt{1-v^2/c^2}}\frac{v}{r}.\]
由以上两式消去$r$可得
\[v = \frac{1}{4\pi\epsilon_0}\frac{Ze^2}{n\hbar} = \frac{Z\alpha c}{n}.\]
考虑到系统在相对论情形下总能量 
\alg{E & = \frac{m_0 c^2}{\sqrt{1-v^2/c^2}} - m_0c^2 - \frac{1}{4\pi\epsilon_0}\frac{Ze^2}{r}\\
& = \frac{m_0 c^2}{\sqrt{1-v^2/c^2}} - m_0c^2 -\frac{m_0v^2}{\sqrt{1-v^2/c^2}}\\
& = m_0c^2(\sqrt{1-\frac{v^2}{c^2}}-1).
}
展开到$o(\frac{v^4}{c^4})$项有
\[E \approx m_0c^2\left(-\frac{v^2}{c^2}-\frac{v^4}{8c^4}\right)
= -\frac{1}{2}m_0c^2\left(\frac{Z\alpha}{n}\right)^2\left[1+\frac{1}{4}\left(\frac{Z\alpha}{n}\right)^2\right]\]
【另解】亦可对动能和势能分别计算,但需始终牢记应计算到$o(\frac{v^4}{c^4})$项:
\[E_k = \frac{m_0 c^2}{\sqrt{1-v^2/c^2}}-m_0c^2 
=m_0c^2\left(1+\frac{v^2}{2c^2}+\frac{(-\frac{1}{2})(-\frac{3}{2})v^4}{2c^4}+o(\frac{v^6}{c^6})\right),\]
\[E_p = -\frac{m_0v^2}{\sqrt{1-v^2/c^2}} = -m_0v^2\left(1+\frac{v^2}{2c^2}+o(\frac{v^4}{c^4})\right).\]
相加得
\[E = m_0c^2\left(-\frac{v^2}{c^2}-\frac{v^4}{8c^4}+o(\frac{v^6}{c^6})\right)\]
亦可得证。

\item 同\,1.45。

\item
由于$\,\lambda=h/p$,故波长相同的粒子动量相同,$p_\gamma/p_{\mathrm{e}}=1$。\par
光子的动能为
\eqa{E_{k\gamma}=\frac{hc}{\lambda}=3.10\,\mathrm{keV}}
电子的动能为
\eqa{E_{k\mathrm{e}}=E_{\mathrm{e}}-E_0=\sqrt{E_0^2+p_{\mathrm{e}}^2c^2}-E_0=\sqrt{E_0^2+\Brak{\dfrac{h}{\lambda}}^2c^2}-E_0=9.40\,\mathrm{eV}}
故二者动能的比值为
\eqa{\frac{E_{k\gamma}}{E_{k\mathrm{e}}}=330}

\item
相对论粒子总能量为
\eqa{E=\frac{m_0c^2}{\sqrt{1-\frac{v^2}{c^2}}}}
电子的动能等于其静质量能,即其总能量等于静质量能的二倍,也就是
\eqa{\frac{m_0c^2}{\sqrt{1-\frac{v^2}{c^2}}}=2m_0c^2}
从中可解出电子速度为
\eqa{v=\frac{\sqrt{3}}{2}c}
依定义,德布罗意波长为
\eqa{\lambda=\frac{h}{p}=\frac{h}{m_0v}\sqrt{1-\frac{v^2}{c^2}}}
其中$\,p=m_0v/\sqrt{1-\frac{v^2}{c^2}}\,$为相对论粒子的动量。代入数据,得
\eqa{\lambda=1.40\,\mathrm{pm}}

\item
由于中子的动能远小于其静质量能$\,E_0=940\,\mathrm{MeV}$,故可以采用经典近似$\,p=\sqrt{2m_0E_k}$,其德布罗意波长为
\eqa{\lambda=\frac{h}{\sqrt{2m_0E_k}}}
代入数据,分别算出两种情况下中子的德布罗意波长
\alg{\lambda_1=&20.2\,\mathrm{pm}\\
\lambda_2=&9.05\,\mathrm{pm}}
由布拉格衍射条件$\,n\lambda=2d\sin\theta$,知
\eqa{\sin\theta=\frac{n\lambda}{2d}}
对于两束中子,其布拉格角分别由下两式给出
\alg{\sin\theta_1=&0.0632n,\quad n=1,\,2,\,\dotsc,\,16;\\
\sin\theta_2=&0.0283n,\quad n=1,\,2,\,\dotsc,\,35}
其中$\,n\,$的范围由$\,\sin\theta<1\,$的条件限定。

\item
按定义,德布罗意频率为
\eqa{\nu=\frac{E}{h}=\frac{E_k+E_0}{h}}
其中$\,E_0\,$为电子的静能。德布罗意波长为
\eqa{\lambda=\frac{h}{p}=\frac{hc}{\sqrt{(E_k+E_0)^2-E_0^2}}}
这里用到了相对论能动量关系
\eqa{pc=\sqrt{E^2-E_0^2}=\sqrt{(E_k+E_0)^2-E_0^2}}
对于动能为$\,1\,\mathrm{eV}$、$100\,\mathrm{eV}\,$及$\,1\,\mathrm{keV}\,$的电子来说,由于$\,E_k\ll E_0=511\,\mathrm{keV}$,这三种情况下可以采用经典近似$\,p=\sqrt{2m_0E_k}$,从而
\eqa{\lambda=\frac{h}{\sqrt{2m_0E_k}}}
代入数据,得到结果如下表
\begin{table}[h]
    \centering
    \begin{tabular}{c|c|c}
    \hline
        \rule{0pt}{13pt}动能 & $\lambda/\mathrm{m}$ & $\nu/\mathrm{Hz}$ \\
        \hline
        \rule{0pt}{13pt}$1\,\mathrm{eV}$ & $1.23\et{-9}$ & $1.23\et{20}$ \\
        $100\,\mathrm{eV}$ & $1.23\et{-10}$ & $1.23\et{20}$ \\
        $1\,\mathrm{keV}$ & $3.88\et{-11}$ & $1.24\et{20}$ \\
        $1\,\mathrm{MeV}$ & $8.73\et{-13}$ & $3.65\et{20}$ \\
        $12\,\mathrm{GeV}$ & $1.03\et{-16}$ & $2.90\et{24}$ \\
    \hline
    \end{tabular}
    \caption{题\,1.45\,表}
\end{table}\par
根据布拉格衍射条件
\eqa{n\lambda=2d\sin\theta}
可知发生衍射的条件为
\eqa{\lambda<2d=4.30\et{-10}\,\mathrm{m}}
故能量为$\,1\,\mathrm{eV}\,$的电子不会发生衍射。(但若$\lambda \ll d$会导致条纹过密,衍射也不明显。故定性来讲衍射最为显著的应是动能为$100\eV$与$1\keV$的电子。)

对$30\du$的布拉格角,需满足$n\lambda = d$。此时按上述五种电子计算,$n$值分别为 0.174, 1.74, 5.54, 246.2, $2.08\ten{6}$。故并不能期望在$30\du$角附近观察到上述五种电子产生的显著衍射条纹。

\item
单缝衍射中央主极大的角展宽为
\eqa{2\theta=2\frac{\lambda}{d}}
其中$\,\lambda\,$为波长,对于电子则是其德布罗意波长,$d\,$为缝宽。代数数据,可得
\eqa{2\theta=0.2\,\mathrm{rad}}

\item
按照定义,粒子的康普顿波长为
\eqa{\lambda_{\mathrm{c}}=\frac{h}{m_0c}}
其中$\,m_0\,$为粒子的静质量。\par
粒子的德布罗意波长为
\eqa{\lambda_{\mathrm{d}}=\frac{h}{p}=\frac{h}{m_0v}\sqrt{1-\dfrac{v^2}{c^2}}}
这里用到了相对论动量
\eqa{p=\frac{m_0v}{\sqrt{1-v^2/c^2}}}
故康普顿波长与德布罗意波长的比值为
\eqa{\frac{\lambda_{\mathrm{c}}}{\lambda_{\mathrm{d}}}=\frac{1}{\sqrt{\dfrac{c^2}{v^2}-1}}}
狭义相对论中,粒子的总能量与静能比值为
\eqa{\frac{E}{E_0}=\frac{1}{\sqrt{1-\dfrac{v^2}{c^2}}}}
容易看出
\eqa{\sqrt{\Brak{\dfrac{E}{E_0}}^2-1}=\sqrt{\dfrac{v^2/c^2}{1-v^2/c^2}}=\dfrac{1}{\sqrt{\Brak{\dfrac{c}{v}}^2-1}}}
显然,当$\,E=\sqrt{2}E_0\,$时,粒子的德布罗意波长等于其康普顿波长。对于电子,此时动能为
\eqa{E_k=E-E_0=(\sqrt{2}-1)E_0=212\,\mathrm{keV}}

\item
由德布罗意波长的定义:
\eqa{\lambda=\frac{h}{p}}
由相对论能动量关系:
\eqa{p=\frac{\sqrt{(E_0+eV)^2-E_0^2}}{c}=\frac{\sqrt{2E_0e}\sqrt{V(1+\frac{e}{2E_0}V)}}{c}=\frac{\sqrt{2E_0e}\sqrt{V_r}}{c}}
其中$\,E_0=511\mathrm{keV}\,$为电子的静能。故
\eqa{\lambda=\frac{hc}{\sqrt{2E_0e}}\frac{1}{V_r}}
代入数据得
\eqa{\lambda=\frac{1.226}{\sqrt{V_r}}\,\mathrm{nm}\cdot\mathrm{V}^{1/2}}
\eqa{V_r=V(1+9.78\et{-7}\,\mathrm{V}^{-1}\,V)}

\end{enumerate}